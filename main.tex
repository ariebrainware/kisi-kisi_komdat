\documentclass[oneside]{book}
\usepackage{xcolor}


\newcommand{\exercisename}{Latihan}
\newcommand{\solutionname}{Solusi}

\definecolor{main}{RGB}{0,120,2}

%% Exercise with counter
\newcounter{exer}[chapter]
\setcounter{exer}{0}
\renewcommand{\theexer}{\thechapter.\arabic{exer}}
\newenvironment{exercise}[1][]{
  \refstepcounter{exer}
  \par\noindent\textbf{\color{main}{\exercisename} \theexer #1 }\rmfamily}{
  \par\ignorespacesafterend}

\newenvironment{solution}{\par\noindent\textbf{\color{main}\solutionname} \em}{\par}

\begin{document}


\chapter{Introduction}

% Pritami Sergio 1-4
\begin{exercise}
  berapa jumlah maksimum karakter atau simbol yang dapat diwakili oleh unicode?
\end{exercise}

\begin{solution}
  Unicode menggunakan 32 bit untuk mewakili simbol atau karakter. Kita dapat mendefinisikan 2^32 simbol atau karakter yang berbeda.
\end{solution}

\vspace{12pt}

\begin{exercise}
  Gambar berwarna menggunakan 16 bit untuk mewakili satu piksel, berapa jumlah maksimum warna berbeda yang dapat direpresentasikan?
\end{exercise}

\begin{solution}
  Dengan 16 bit,bisa mewakili hingga 2^16 warna berbeda
\end{solution}

\vspace{12pt}

\begin{exercise}
  asumsikan enam perangkat disusun dalam topologi mesh berapa banyak kabel yang dibutuhkan? dan berapa banyak port yang dibutuhkan untuk setiap perangkat ?
\end{exercise}

\begin{solution}
  Untuk menghitung kabel dan ports yang digunakan dalam topologi mesh, harus menggunakan rumus N*N-1/2 jika yang digunakan 6 perangkat maka 6*6-1/2 dan untuk port n-1 jika 6 perangkat maka 6 - 1 = 5 port
\end{solution}

\vspace{12pt}

\begin{exercise}
  Dari empat jaringan berikut, diskusikan konsekuensinya jika koneksi gagal
  a. 5 Perangkat dengan topologi mesh
  b. 5 Perangkat dengan topologi Star
  c. 5 Perangkat dengan topologi bus
  d. 5 Perangkat dengan topologi ring
\end{exercise}

\begin{solution}
  a. Jika satu koneksi gagal, koneksi lainnya akan tetap berfungsi.
  b. Perangkat lain masih dapat mengirim data melalui hub; tidak akan ada bisa mengakses pada perangkat yang koneksinya gagal ke hub
  c. Semua transmisi berhenti jika kegagalan ada di bus. Jika drop-line gagal, hanya perangkat yang ada pada jalur tersebut yang tidak dapat beroperasi
  d. Koneksi yang gagal dapat mematikan seluruh jaringan kecuali jika ada cincin ganda atau ada mekanisme by-pass
\end{solution}

\vspace{12pt}

% Muhammad Adi Syahputra 5-8
\begin{exercise}
  Contoh soal 5
\end{exercise}

\begin{solution}
  Contoh solusi
\end{solution}

\vspace{12pt}

\begin{exercise}
  Contoh soal
\end{exercise}

\begin{solution}
  Contoh solusi
\end{solution}

\vspace{12pt}

\begin{exercise}
  Contoh soal
\end{exercise}

\begin{solution}
  Contoh solusi
\end{solution}

\vspace{12pt}

\begin{exercise}
  Contoh soal
\end{exercise}

\begin{solution}
  Contoh solusi
\end{solution}

\vspace{12pt}

% Amelia Marta Dilova 9-12
\begin{exercise}
  Contoh soal 9
\end{exercise}

\begin{solution}
  Contoh solusi
\end{solution}

\vspace{12pt}

\begin{exercise}
  Contoh soal
\end{exercise}

\begin{solution}
  Contoh solusi
\end{solution}

\vspace{12pt}

\begin{exercise}
  Contoh soal
\end{exercise}

\begin{solution}
  Contoh solusi
\end{solution}

\vspace{12pt}

\begin{exercise}
  Contoh soal 12
\end{exercise}

\begin{solution}
  Contoh solusi
\end{solution}

\chapter{Network Model}

% Muhammad Riyadhtul Akbar 1-3
\begin{exercise}
  Bagaimana OSI dan ISO terkait satu sama lain?
\end{exercise}

\begin{solution}
  OSI dipakai sebagai model untuk membentuk standar umum jaringan komputer sehingga menunjang antaroperasi perangkat jaringan dari pemasok perangkat jaringan yang berbeda. ISO ialah organisasi yg dibentuk untuk membuat dan memperkenalkan standardisasi internasional untuk apa saja
\end{solution}

\vspace{12pt}

\begin{exercise}
  Cocokkan yang berikut ini dengan satu atau lebih lapisan model OSI:
  sebuah. 
  a. Penentuan rute
  b. Alur kontrol
  c. Antarmuka ke media transmisi
  d. Menyediakan akses untuk pengguna akhir
\end{exercise}

\begin{solution}
  a. Network Layer Untuk mendefinisikan alamat-alamat IP dan menyediakan fungsi routing sehingga paket dapat dikirim keluar dari segment network lokal ke suatu tujuan yang berada pada suatu network lain. Contoh protocol yang digunakan seperti IP
  b. Application Layer Merupakan layer dimana terjadi interaksi antarmuka end user dengan aplikasi yang bekerja menggunakan fungsionalitas jaringan, melakukan pengaturan bagaimana aplikasi bekerja menggunakan resource jaringan, untuk kemudian memberika pesan ketika terjadi kesalahan. Beberapa service dan protokol yang berada di layer ini misalnya HTTP, FTP, SMTP, dll.
  c. Presentation Layer adalah presentation layer, dimana mempunyai fungsi untuk mentranslasikan format data yang akan ditransmisikan oleh aplikasi melalui jaringan, ke dalam format yang dapat ditransmisikan oleh sebuah jaringan.
  d. Session Layer juga mengatur checkpoint selama proses transfer, apabila session terganggu, perangkat dapat melanjutkan transfer data dari checkpoint terakhir.
\end{solution}

\vspace{12pt}

\begin{exercise}
  Cocokkan yang berikut ini dengan satu atau lebih lapisan model OSI:
  sebuah. 
  a. Pengiriman pesan proses-ke-proses yang andal
  b. Pemilihan rute
  c. Mendefinisikan frame
  d. Menyediakan layanan pengguna seperti email dan transfer file
  e. Transmisi aliran bit melintasi media fisik
\end{exercise}

\begin{solution}
  a. Transport Layer bertugas untuk mengambil data dari session layer dan membaginya ke bagian-bagian paket data yang lebih kecil. Kemudian data yang sudah sampai ke tujuan akan digabung kembali.
  b.  Network Layer Layer ketiga ini memiliki dua fungsi utama yaitu menemukan jalur terbaik pada jaringan (routing) untuk proses pertukaran data.
  c. Pada data-link layer memiliki tugas untuk menentukan setiap bit data dikelompokkan menjadi format yang disebut dengan frame. Pada level ini juga terjadi koreksi kesalahan, flow control, pengalamatan hardware atau perangkat keras (seperti halnya pada MAC Address (Media Access Control Address)). 
  d. Presentation Layer saat data mulai ditransfer, dan bertindak ketika sebuah komputer menerima paket data disebut dengan nama Presentation Layer. Funsi utama dari lapisan layer presentation ini adalah menteranslate data yang akan ditransmisikan dari dan menuju sebuah application (aplikasi).
  e. Physical Layer Tentu saja hal itu dilakukan dari physical layer pengirim dan ditujukan kepada physical layer penerima.Nantinya di sini data juga akan ditransmisikan dengan memakai jenis sinyal yang telah didukung media fisik.
\end{solution}

\vspace{12pt}

% Muhammad Rizky Fadillah 4-7
\begin{exercise}
  Contoh soal 4
\end{exercise}

\begin{solution}
  Contoh solusi
\end{solution}

\vspace{12pt}

\begin{exercise}
  Contoh soal
\end{exercise}

\begin{solution}
  Contoh solusi
\end{solution}

\vspace{12pt}

\begin{exercise}
  Contoh soal
\end{exercise}

\begin{solution}
  Contoh solusi
\end{solution}

\vspace{12pt}

\begin{exercise}
  Contoh soal
\end{exercise}

\begin{solution}
  Contoh solusi
\end{solution}

\vspace{12pt}

% Rizky Sandiary 8-11
\begin{exercise}
Bagaimana OSI dan ISO terkait satu sama lain?
\end{exercise}

\begin{solution}
ISO adalah organisasi (Organisasi Standar Internasional), dan OSI (Interkoneksi Sistem Terbuka) adalah modelnya.
\end{solution}

\vspace{12pt}

\begin{exercise}
Misalkan komputer mengirimkan paket pada lapisan jaringan ke komputer lain di suatu tempat di Internet. Alamat tujuan logis dari paket rusak. Apa yang terjadi pada paket? Bagaimana komputer sumber dapat mengetahui situasinya?
\end{exercise}

\begin{solution}
Sebelum menggunakan alamat tujuan di perantara atau node tujuan, paket melewati pemeriksaan kesalahan yang dapat membantu node menemukan korupsi (dengan probabilitas tinggi) dan membuang paket. Biasanya protokol lapisan atas akan menginformasikan sumber untuk mengirim ulang paket.
\end{solution}

\vspace{12pt}

\begin{exercise}
Jika lapisan data link dapat mendeteksi kesalahan antar hop, mengapa menurut Anda kita memerlukan mekanisme pemeriksaan lain di lapisan transport?
\end{exercise}

\begin{solution}
Kesalahan antar node dapat dideteksi oleh kontrol lapisan data link, tetapi kesalahan pada node (antara port input dan port output) dari node tidak dapat dideteksi oleh lapisan data link
\end{solution}

\vspace{12pt}

\begin{exercise}
Misalkan sebuah komputer mengirimkan sebuah frame ke komputer lain pada topologi bus LAN. Alamat tujuan fisik frame rusak selama transmisi. Apa yang terjadi pada bingkai? Bagaimana pengirim dapat diberitahu tentang situasinya?
\end{exercise}

\begin{solution}
Jika alamat tujuan yang rusak tidak cocok dengan alamat stasiun mana pun di jaringan, paket akan hilang. Jika alamat tujuan yang rusak cocok dengan salah satu stasiun, frame dikirimkan ke stasiun yang salah. Namun, dalam kasus ini, mekanisme pendeteksian kesalahan, yang tersedia di sebagian besar protokol tautan data, akan menemukan kesalahan dan membuang bingkai. Dalam kedua kasus, sumber entah bagaimana akan diinformasikan menggunakan salah satu mekanisme kontrol tautan data.
\end{solution}

\chapter{Data and Signals}

% Dimas Yediberto Luciano Dien 1-3
\begin{exercise}
  Contoh soal 1
\end{exercise}

\begin{solution}
  Contoh solusi
\end{solution}

\vspace{12pt}

\begin{exercise}
  Contoh soal
\end{exercise}

\begin{solution}
  Contoh solusi
\end{solution}

\vspace{12pt}

\begin{exercise}
  Contoh soal
\end{exercise}

\begin{solution}
  Contoh solusi
\end{solution}

\vspace{12pt}

% Karel Chavez H 4-6
\begin{exercise}
Berapakah lebar bandwidth suatu sinyal yang dapat diuraikan menjadi lima gelombang sinus dengan frekuensi 0, 20, 50, 100, dan 200 Hz? Semua amplitudo puncak adalah sama. Gambarkan bandwidthnya.
\end{exercise}

\begin{solution}
220 ns = 220 x 10 -9 s = apakah bandwidth suatu sinyal dapat diuraikan menjadi lima gelombang sinus dengan frekuensi 0, 20, 50, 100, dan 200 Hz? Semua amplitudo puncak adalah sama.
\end{solution}

\vspace{12pt}

\begin{exercise}
Sinyal composite signal dengan bandwidth 2000 Hz terdiri dari dua gelombang sinus. Yang pertama memiliki frekuensi 100 Hz dengan amplitudo maksimum 20 V; yang kedua memiliki amplitudo maksimum 5 V. Gambarkan bandwidthnya.
\end{exercise}

\begin{solution}
bandwidth = Fh - Fl, bandwith = 2000 , Terendah = 100 , Tertinggi = 2100 Fh - Fl = 2100 - 100 ,bandwith = 2000
\end{solution}

\vspace{12pt}

\begin{exercise}
Sinyal manakah yang memiliki bandwidth lebih lebar, gelombang sinus dengan frekuensi 100 Hz atau gelombang sinus dengan frekuensi 200 Hz?
\end{exercise}

\begin{solution}
setiap sinyal adalah sinyal sederhana dalam hal ini. Bandwidth dari sinyal sederhana adalah 0.jadi, bandwidth dari kedua sinyal r sama.
\end{solution}

\vspace{12pt}

% Muhammad Arie Setya Putra Pala 7-9
\begin{exercise}
\\
Apa hubungan teorema Nyquist dengan komunikasi?
\end{exercise}

\begin{solution}
\\
Teorema Nyquist-Shannon juga dikenal sebagai teorema pengambilan sampel adalah ketentuan fisik mendasar untuk komunikasi di mana sinyal kontinu dalam waktu terkait dengan sinyal diskrit dalam waktu. Ini pada dasarnya menetapkan jumlah pengambilan sampel minimum yang memungkinkan urutan diskrit untuk menangkap semua sinyal kontinu.
\end{solution}

\vspace{12pt}

\begin{exercise}
\\
Apa hubungan kapasitas Shannon dengan komunikasi?
\end{exercise}

\begin{solution}
\\
Batas Shannon atau kapasitas Shannon dari saluran komunikasi mengacu pada tingkat maksimum data bebas kesalahan yang secara teoritis dapat ditransfer melalui saluran jika tautan mengalami kesalahan transmisi data acak, untuk tingkat kebisingan tertentu.
\end{solution}

\vspace{12pt}

\begin{exercise}
\\
Mengapa sinyal optik yang digunakan pada kabel serat optik memiliki panjang gelombang yang sangat pendek?
\end{exercise}

\begin{solution}
Optical signals have very high frequencies. A high frequency means a short wave length because the wave length is inversely proportional to the frequency.
\end{solution}

\vspace{12pt}

% Ricky 10-12
\begin{exercise}
\\
Bisakah kita mengatakan jika suatu sinyal periodik atau nonperiodik hanya dengan melihat frekuensinya petak domain ? bagaimana ?
\end{exercise}

\begin{solution}
\\
  bisa, karena sinyal periodik dapat dilihat dari frekuensinya yang memiliki periode waktu dasar berulang pada interval waktu yang teratur sedangkan sinyal non-periodik itu acak dan tidak dapat di definisi seperti pada gelombang sinus atau gelombang kosinus.
\end{solution}

\vspace{12pt}

\begin{exercise}
\\
Apakah plot domain frekuensi dari sinyal suara itu diskrit atau kontinu?
\end{exercise}

\begin{solution}
\\
Domain frekuensi sinyal suara biasanya kontinu karena suara adalah sinyal nonperiodik.
\end{solution}

\vspace{12pt}

\begin{exercise}
\\
Apakah plot domain frekuensi dari sistem alarm itu diskrit atau kontinu?
\end{exercise}

\begin{solution}
\\
Sistem alarm biasanya periodik. Oleh karena itu, plot domain frekuensinya adalah diskrit.
\end{solution}

\vspace{12pt}

% Fajar Bimantara 13-16
\begin{exercise}
  Kami mengirim sinyal suara dari mikrofon ke perekam. Apakah ini transmisi baseband atau broadband?
\end{exercise}

\begin{solution}
  broadband
\end{solution}

\vspace{12pt}

\begin{exercise}
  Kami mengirim sinyal digital dari satu stasiun di LAN ke stasiun lain. Apakah ini transmisi baseband atau broadband?
\end{exercise}

\begin{solution}
  broadband
\end{solution}

\vspace{12pt}

\begin{exercise}
  Kami memodulasi beberapa sinyal suara dan mengirimkannya melalui udara. Apakah ini pita dasar? atau transmisi broadband? Latihan
\end{exercise}

\begin{solution}
  pita dasar
\end{solution}

\vspace{12pt}

\begin{exercise}
  Mengingat frekuensi yang tercantum di bawah ini, hitung periode yang sesuai.
  sebuah. 
  a. 24Hz
  b. 8 MHz
  c. 140 KHz
\end{exercise}

\begin{solution}
  a. 12491352.42m
  b. 37.462818375m
  c. 2140.73m
\end{solution}

\vspace{12pt}

% Julicko Pratama Putra 17-19
\begin{exercise}
  Contoh soal 17
\end{exercise}

\begin{solution}
  Contoh solusi
\end{solution}

\vspace{12pt}

\begin{exercise}
  Contoh soal
\end{exercise}

\begin{solution}
  Contoh solusi
\end{solution}

\vspace{12pt}

\begin{exercise}
  Contoh soal
\end{exercise}

\begin{solution}
  Contoh solusi
\end{solution}

\vspace{12pt}

% Nadjamudin Beda 20-23
\begin{exercise}
  Contoh soal 20
\end{exercise}

\begin{solution}
  Contoh solusi
\end{solution}

\vspace{12pt}

\begin{exercise}
  Contoh soal
\end{exercise}

\begin{solution}
  Contoh solusi
\end{solution}

\vspace{12pt}

\begin{exercise}
  Contoh soal
\end{exercise}

\begin{solution}
  Contoh solusi
\end{solution}

\vspace{12pt}

\begin{exercise}
  Contoh soal
\end{exercise}

\begin{solution}
  Contoh solusi
\end{solution}

\chapter{Digital Transmission}

% Pritami Sergio 1-4
\begin{exercise}
  Contoh soal 1
\end{exercise}

\begin{solution}
  Contoh solusi
\end{solution}

\vspace{12pt}

\begin{exercise}
  Contoh soal
\end{exercise}

\begin{solution}
  Contoh solusi
\end{solution}

\vspace{12pt}

\begin{exercise}
  Contoh soal
\end{exercise}

\begin{solution}
  Contoh solusi
\end{solution}

\vspace{12pt}


\begin{exercise}
  Contoh soal
\end{exercise}

\begin{solution}
  Contoh solusi
\end{solution}

\vspace{12pt}

% Muhammad Adi Syahputra 5-8
\begin{exercise}
  Contoh soal 5
\end{exercise}

\begin{solution}
  Contoh solusi
\end{solution}

\vspace{12pt}

\begin{exercise}
  Contoh soal
\end{exercise}

\begin{solution}
  Contoh solusi
\end{solution}

\vspace{12pt}

\begin{exercise}
  Contoh soal
\end{exercise}

\begin{solution}
  Contoh solusi
\end{solution}

\vspace{12pt}

\begin{exercise}
  Contoh soal
\end{exercise}

\begin{solution}
  Contoh solusi
\end{solution}

\vspace{12pt}

% Amelia Marta Dilova 9-12
\begin{exercise}
  Contoh soal 9
\end{exercise}

\begin{solution}
  Contoh solusi
\end{solution}

\vspace{12pt}

\begin{exercise}
  Contoh soal
\end{exercise}

\begin{solution}
  Contoh solusi
\end{solution}

\vspace{12pt}

\begin{exercise}
  Contoh soal
\end{exercise}

\begin{solution}
  Contoh solusi
\end{solution}

\vspace{12pt}

\begin{exercise}
  Contoh soal
\end{exercise}

\begin{solution}
  Contoh solusi
\end{solution}

\vspace{12pt}

% Muhammad Riyadhtul Akbar 13-16
\begin{exercise}
  Hitung nilai laju sinyal untuk setiap kasus pada Gambar 4.2 jika laju data 1 Mbps dan c = 1/2.
\end{exercise}

\begin{solution}
  Sebuah sinyal dengan tingkat L benar-benar dapat membawa bit
  log2L per tingkat. Jika setiap tingkat sesuai dengan salah satu
  elemen sinyal dan kita asumsikan rata-ratanya jika (c = 1/2), maka
  kita harus
  Nmax = 1/c x B x r = 2 x B x log2L
\end{solution}

\vspace{12pt}

\begin{exercise}
  Dalam transmisi digital, jam pengirim 0,2 persen lebih cepat dari jam penerima. Berapa bit ekstra per detik yang dikirim pengirim jika kecepatan data 1 Mbps?
\end{exercise}

\begin{solution}
  Pada 1 Mbps, penerima menerima 1.001.000 bps alih-alih 
  1.000.000 bps.

\end{solution}

\vspace{12pt}

\begin{exercise}
  Gambarkan grafik skema NRZ-L menggunakan masing-masing aliran data berikut,
  dengan asumsi bahwa signa11evel terakhir telah positif. Dari grafik, tebak
  bandwidth untuk skema ini menggunakan jumlah rata-rata perubahan level sinyal.
  Bandingkan tebakan Anda dengan entri yang sesuai pada Tabel 4.1.
  sebuah. 
  a. 00000000
  b. 11111111
  c. 01010101
  d. 00110011
\end{exercise}

\begin{solution}
  a. S = c x N x -
  = - x 100.000 x -1 = 50.000 =50 kbaud
  b. S = c x N x -
  = - x 100.000 x -1 = 50.000 =50 kbaud
  c. S = c x N x -
  = - x 100.000 x -1 = 50.000 =50 kbaud
  d. S = c x N x -
  = - x 100.000 x -1 = 50.000 =50 kbaud
\end{solution}

\vspace{12pt}

\begin{exercise}
  Ulangi Latihan 15 untuk skema NRZ-I
\end{exercise}

\begin{solution}
  Tingkat sinyal rata-rata adalah S = NI2 = 500 kbaud. Bandwidth minimum untuk baud rata-rata ini
  kecepatannya adalah Bnlin = S = 500 kHz
\end{solution}

\vspace{12pt}



% Muhammad Rizky Fadillah 17-20
\begin{exercise}
  Contoh soal 17
\end{exercise}

\begin{solution}
  Contoh solusi
\end{solution}

\vspace{12pt}

\begin{exercise}
  Contoh soal
\end{exercise}

\begin{solution}
  Contoh solusi
\end{solution}

\vspace{12pt}

\begin{exercise}
  Contoh soal
\end{exercise}

\begin{solution}
  Contoh solusi
\end{solution}

\vspace{12pt}

\begin{exercise}
  Contoh soal
\end{exercise}

\begin{solution}
  Contoh solusi
\end{solution}

\chapter{Analog Transmission}

% Rizky Sandiary 1-4
\begin{exercise}
  Calculate the baud rate for the given bit rate and type of modulation.
  \begin{itemize}
    \item[a.] 2000 bps, FSK
    \item[b.] 4000 bps, ASK
  \end{itemize}
\end{exercise}

\begin{solution}
  We use the formula $S = (1/r) \times N$, but first we need to calculate the value of r for each case.
  \begin{itemize}
    \item[a.] $r = log_22 = 1 \quad \rightarrow \quad S = (1/1) \times (2000 \textnormal{ bps}) = 2000 \textnormal{ baud}$
    \item[b.] 
  \end{itemize}
\end{solution}

\vspace{12pt}

\begin{exercise}
\\
Temukan bandwidth untuk situasi berikut jika kita perlu memodulasi suara 5-KHz.
a. AM
b. PM (set =5)
c. PM (set =1)
\end{exercise}

\begin{solution}
\\
Mengingat frekuensi sinyal suara -

=5kHz

a) Bandwidth modulasi amplitudo Bam=2B
=2 × 5kHz
=10kHz

b) Bandwidth yang dibutuhkan untuk modulasi fase
Bpm=2(1+β)B 
Sekarang, menggantikan nilai-nilai,
=2(1+3) × 5kHz
=40kHz

c) Bandwidth yang dibutuhkan untuk modulasi fase,
Bpm=2(1+β)B 
Sekarang, menggantikan nilai-nilai,
=2(1+1) × 5kHz
=20kHz
\end{solution}

\vspace{12pt}

\begin{exercise}
\\
Saluran telepon memiliki bandwidth 4 KHz. Berapa jumlah bit maksimum yang kami miliki?
dapat mengirim menggunakan masing-masing teknik berikut? Misalkan d = O
sebuah. 
A. ASK
B. QPSK
C. 16-QAM
D. 64-QAM
\end{exercise}

\begin{solution}
Kami menggunakan rumus N = [1/(1 + d)] × r × B, tetapi pertama-tama kita perlu menghitung nilai r untuk setiap kasus. sebuah. 
A. r = log22 = 1 →N= [1/(1 + 0)] × 1 × (4 KHz) = 4 kbps 
B. r = log24=2 →N = [1/(1 + 0)] × 2 × (4 KHz) = 8 kbps 
C. r = log216= 4 →N = [1/(1 + 0)] × 4 × (4 KHz) = 16 kbps 
D. r = log264= 6 →N = [1/(1 + 0)] × 6 × (4 KHz) = 24 kbpsQ19.
\end{solution}

\vspace{12pt}

\begin{exercise}
Sebuah perusahaan kabel menggunakan salah satu saluran TV kabel (dengan bandwidth 6 MHz) untuk menyediakan komunikasi digital bagi setiap penduduk. Berapa kecepatan data yang tersedia untuk setiap penduduk jika perusahaan menggunakan teknik 64-QAM?
\end{exercise}

\begin{solution}
Kita dapat menggunakan rumus: N = [1/(1 + d)] × r × B = 1 × 6 × 6 MHz = 36 Mbps
\end{solution}

\vspace{12pt}

% Dimas Yediberto Luciano Dien 5-8
\begin{exercise}
  Contoh soal 5
\end{exercise}

\begin{solution}
  Contoh solusi
\end{solution}

\vspace{12pt}

\begin{exercise}
  Contoh soal
\end{exercise}

\begin{solution}
  Contoh solusi
\end{solution}

\vspace{12pt}

\begin{exercise}
  Contoh soal
\end{exercise}

\begin{solution}
  Contoh solusi
\end{solution}

\vspace{12pt}

\begin{exercise}
  Contoh soal
\end{exercise}

\begin{solution}
  Contoh solusi
\end{solution}

\vspace{12pt}

% Karel Chavez H 9-12
\begin{exercise}
Sebuah perusahaan memiliki media dengan bandwidth 1-MHz (lowpass). Korporasi perlu membuat 10 saluran independen terpisah yang masing-masing mampu mengirim setidaknya 10 Mbps. Perusahaan telah memutuskan untuk menggunakan teknologi QAM. Berapa jumlah bit minimum per baud untuk setiap saluran? Berapa jumlah titik dalam diagram konstelasi untuk setiap saluran? Misalkan d = O.
\end{exercise}

\begin{solution}
Pertama, kami menghitung bandwidth untuk setiap saluran = (1 MHz) / 10 = 100 KHz. Kami kemudian menemukan nilai r untuk setiap saluran: B = (1 + d) × (1/r) × (N) →r = N / B →r = (1 Mbps/100 KHz) = 10 Kemudian kita dapat menghitung jumlah level: L = 2r = 210 = 1024. Ini berarti bahwa kita memerlukan teknik 1024-QAM untuk mencapai kecepatan data ini.
\end{solution}

\vspace{12pt}

\begin{exercise}
Berapa bit per baud yang dapat kita kirim dalam setiap kasus berikut jika konstelasi sinyal memiliki salah satu dari jumlah titik berikut?
\begin{itemize}
\item[a.] 2
\item[b.] 4
\item[c.] 16
\item[d.] 1024
\end{itemize}
\end{exercise}

\begin{solution}
  \begin{itemize}
    \item[a.] $log_22 = 1$
    \item[b.] $log_24 = 2$
    \item[c.] $log_216 = 4$
    \item[d.] $log_21024 = 10 $
  \end{itemize}
\end{solution}

\vspace{12pt}

\begin{exercise}
Hitung bit rate untuk baud rate yang diberikan dan jenis modulasi
  \begin{itemize}
    \item[a.] 1000 baud, FSK
    \item[b.] 1000 baud, ASK
    \item[c.] 1000 baud, BPSK
    \item[d.] 1000 baud, 16-QAM
  \end{itemize}
\end{exercise}

\begin{solution}
\begin{itemize}
    \item[a.] Modulator yang dimaksud adalah FSK dan r = 1 Jadi, baud rate = 2000bps/1 baud rate = 2000baud
    \item[b.] Modulator yang digunakan adalah ASK dan r = 1, baud rate = 4000bps/1 baud rate = 4000baud
    \item[c.] Modulator yang digunakan adalah QPSK/4-PSK dan r = 2, Jadi, baud rate = 6000bps/2 baud rate = 3000baud
    \item[d.] Modulator yang digunakan adalah 64-QAM, baud rate = 36000/6 baud rate = 6000baud
  \end{itemize}
\end{solution}

\vspace{12pt}

\begin{exercise}
Berapa jumlah bit per baud untuk teknik berikut?
  \begin{itemize}
    \item[a.] ASK dengan empat amplitudo berbeda
    \item[b.] FSK dengan 8 frekuensi berbeda
    \item[c.] PSK dengan empat fase berbeda
    \item[d.] QAM dengan konstelasi 128 point
  \end{itemize}
\end{exercise}

\begin{solution}
Kami menggunakan rumus $r = log_2L untuk menghitung nilai r untuk setiap kasus.$
  \begin{itemize}
    \item[a.] $log_24 = 2$
    \item[b.] $log_28 = 3$
    \item[c.] $log_24 = 2$
    \item[d.] $log_2128 = 7$
  \end{itemize}
\end{solution}

\chapter{Bandwidth Utilization: Multiplexing and Spreading}

% Muhammad Arie Setya Putra Pala 1-4
\begin{exercise}
\\
Jelaskan tujuan dari multiplexing
\end{exercise}

\begin{solution}
\\
Tujuan multiplexing adalah untuk memungkinkan sinyal ditransmisikan lebih efisien melalui saluran komunikasi tertentu, sehingga mengurangi biaya transmisi.
\end{solution}

\vspace{12pt}

\begin{exercise}
\\
Sebutkan tiga teknik multiplexing utama yang disebutkan dalam bab ini.
\end{exercise}

\begin{solution}
\\
frequency-division multiplexing (FDM), wave-division multiplexing (WDM), and time-division multiplexing (TDM).
\end{solution}

\vspace{12pt}

\begin{exercise}
\\
Bedakan antara tautan dan saluran dalam multiplexing.
\end{exercise}

\begin{solution}
\\
Dalam multiplexing, kata link mengacu pada jalur fisik. Kata saluran mengacu pada bagian dari tautan yang membawa transmisi antara sepasang garis tertentu. Satu tautan dapat memiliki banyak (n) saluran.
\end{solution}

\vspace{12pt}

\begin{exercise}
\\
Manakah dari tiga teknik multiplexing yang digunakan untuk menggabungkan sinyal analog?
Manakah dari tiga teknik multiplexing yang digunakan untuk menggabungkan sinyal digital?
\end{exercise}

\begin{solution}
\\
FDM dan WDM digunakan untuk menggabungkan sinyal analog; bandwidth dibagi. TDM digunakan untuk menggabungkan sinyal digital; waktunya dibagi.
\end{solution}

\vspace{12pt}

% Ricky 5-7
\begin{exercise}
\\
Tentukan hierarki analog yang digunakan oleh perusahaan telepon dan buat daftar level hierarki yang berbeda.
\end{exercise}

\begin{solution}
\\
Hirarki analog menggunakan saluran suara (4 KHz), grup (48 KHz), grup super (240 KHz), grup master (2,4 MHz), dan grup jumbo (15,12 MHz). \\
Struktur analog tertentu menggunakan saluran distribusi kata. (kelas, kelompok, kelas reli, jumbogroup).
\end{solution}

\vspace{12pt}

\begin{exercise}
\\
Tentukan hierarki analog yang digunakan oleh perusahaan telepon dan buat daftar level hierarki yang berbeda.
\end{exercise}

\begin{solution}
\\
Hirarki analog menggunakan saluran suara (4 KHz), grup (48 KHz), grup super (240 KHz), grup master (2,4 MHz), dan grup jumbo (15,12 MHz). \\
Struktur analog tertentu menggunakan saluran distribusi kata. (kelas, kelompok, kelas reli, jumbogroup).
\end{solution}

\vspace{12pt}

\begin{exercise}
\\
Manakah dari tiga teknik multiplexing yang umum untuk link serat optik? Jelaskan alasannya.
\end{exercise}

\begin{solution}
\\
WDM umum untuk multiplexing sinyal optik karena memungkinkan multiplexing sinyal dengan frekuensi yang sangat tinggi.
\end{solution}

\vspace{12pt}

% Fajar Bimantara 8-11
\begin{exercise}
  Bedakan antara TDM bertingkat, TDM banyak slot, dan TDM isi pulsa.
\end{exercise}

\begin{solution}
  TDM (Time Division Multiplexing) dan FDM (Frequency Division Multiplexing) adalah dua teknik multiplexing. Perbedaan umum antara TDM dan FDM adalah bahwa TDM berbagi skala waktu untuk sinyal yang berbeda; Sedangkan FDM berbagi skala frekuensi untuk sinyal yang berbeda. Sebelum memahami kedua istilah ini secara mendalam, mari kita memahami istilah multiplexing. Multiplexing adalah teknik di mana beberapa sinyal secara bersamaan dikirim melalui satu data link. Sistem multiplexed melibatkan sejumlah perangkat yang berbagi kapasitas satu tautan, sehingga tautan (jalur) dapat memiliki banyak saluran.
\end{solution}

\vspace{12pt}

\begin{exercise}
  Bedakan antara TDM sinkron dan statistik.
\end{exercise}

\begin{solution}
  kasus  TDM  sinkron,  tetapi  dpt  dimanfaatkan  sepenuhnya  oleh  slot  menuju  waktu  yang  lebih  sedikit  dan  untuk  mentransmisikan  pemanfaatan  bandwidth  dari  media.  Dalam  kasus  TDM  statistik, data  di  setiap  slot  harus  memiliki  alamat,  yang  mengidentifikasi  sumber  data.  Karena  data  yang  tiba  dari  dan  didistribusikan  ke  garis  I  /  O  tak  terduga, 
\end{solution}

\vspace{12pt}

\begin{exercise}
  Mendefinisikan spread spectrum dan tujuannya. Sebutkan dua teknik spread spectrum yang dibahas dalam bab ini.
\end{exercise}

\begin{solution}
  Teknologi Spectrum Merupakan teknologi dengan teknik komunikasi yang menitik beratkan pada penggunaan bandwith dan power peak. Teknik ini juga lebi bnyak menggunakan modulasi LAN nirkabel, dan memiliki bentuk sinyal seperti sinyal noise tujuannya dikirimkan dengan menggunakan narrowband carrier signal dan menyebarkan sinyal itu pada kisaran frekuensi yang jauh lebih besar. Sebagai contoh, kita mungkin menggunakan 1 MHz pada 10 Watt dengan narrowband, namun pada spread spectrum kita dapat menggunakan 20 MHz pada 100 mW.
  1.DSSS (Direct Sequence Spread Spectrum)
  Merupakan jenis spread spectrum yang paling luas dikenal dan paling banyak digunakan, karena sistem ini dikenal paling mudah implementasinya dan memiliki data rate yang tinggi.
  2. FHSS (Frequency Hopping Spread Spectrum)
  Merupakan frekuensi hop dengan system pembaan dengan cara melompat dengan urutan yang bersifat pseudorandom. Pseudorandom adalah daftar frekuensi kemana arah frekuensi akan melompat dalam satu interval.
\end{solution}

\vspace{12pt}

\begin{exercise}
  Definisikan FHSS dan jelaskan bagaimana ia mencapai penyebaran bandwidth.
\end{exercise}

\begin{solution}
  Bandwidth merupakan sebuah kapasitas yang bisa dipakai di kabel ethernet supaya bisa dilewati oleh trafik paket data dengan maksimal tertentu. Adapun definisi lain dari bandwidth internet yaitu jumlah konsumsi transfer data yang dihitung di dalam satuan waktu bit per second atau bps. elakukan proses pengiriman dan juga penerimaan data hanya dalam hitungan detik. Ada juga istilah bandwidth analog. Dimana bandwidth analog ini berarti sebuah perbedaan antara frekuensi yang paling rendah dan frekuensi yang paling tinggi di suatu rentang frekuensi yang bisa diukur menggunakan satuan Hertz (Hz) yang bertujuan untuk mengetahui data ataupun informasi yang bisa ditransmisikan di suatu waktu.
\end{solution}

\vspace{12pt}

% Julicko Pratama Putra 12-14
\begin{exercise}
  Contoh soal 12
\end{exercise}

\begin{solution}
  Contoh solusi
\end{solution}

\vspace{12pt}

\begin{exercise}
  Contoh soal
\end{exercise}

\begin{solution}
  Contoh solusi
\end{solution}

\vspace{12pt}

\begin{exercise}
  Contoh soal
\end{exercise}

\begin{solution}
  Contoh solusi
\end{solution}

\vspace{12pt}

% Nadjamudin Beda 15-18
\begin{exercise}
  Contoh soal 15
\end{exercise}

\begin{solution}
  Contoh solusi
\end{solution}

\vspace{12pt}

\begin{exercise}
  Contoh soal
\end{exercise}

\begin{solution}
  Contoh solusi
\end{solution}

\vspace{12pt}

\begin{exercise}
  Contoh soal
\end{exercise}

\begin{solution}
  Contoh solusi
\end{solution}

\vspace{12pt}

\begin{exercise}
  Contoh soal
\end{exercise}

\begin{solution}
  Contoh solusi
\end{solution}

\end{document}

